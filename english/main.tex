% CV - Pablo Agustin Lobo 

% (c) 2018 Pablo Agustin Lobo 
%
\documentclass[a4paper,10pt]{article}
\usepackage{ucs}
\usepackage{multicol}
\usepackage{ragged2e}
\usepackage{lipsum}
\usepackage{array}
\usepackage[utf8x]{inputenc}
\usepackage[margin=0.5in]{geometry}
\raggedbottom
\raggedright
\setlength{\tabcolsep}{0in}
\pagestyle{empty}
\begin{document}

{\LARGE \textbf{Pablo Agustín Lobo} }\\
\medskip \hrule height 1pt \smallskip

\begin{tabular*}{7.25in}{l@{\extracolsep{\fill}}r}
& \textit{plobo@itba.edu.ar · https://www.linkedin.com/in/pablo-a-lobo/ · +(54911) 5116 8801}
\end{tabular*}

{\Large \textbf{Personal Details}}
  \vspace{-2.5mm}
    \begin{itemize}
      \setlength{\itemsep}{3pt}
      \setlength{\parskip}{0pt}
      \setlength{\parsep}{0pt}
      \begin{multicols}{2}
        \item \textbf{DNI:} 38069162.
        \item \textbf{Date of birth:} 17/02/1994.
        \item \textbf{Place of birth:} CABA, Argentina.
        \item \textbf{Nationality:} Argentine.
        \item \textbf{Citizenship/Passport:} Argentine, Spanish.
        \item \textbf{Marital Status:} Single.
      \end{multicols}
    \end{itemize}
  \vspace{-1mm}

\medskip

{\Large \textbf{Education}}
  \vspace{-1.75mm}
\begin{itemize}
  \setlength{\itemsep}{2pt}
  \setlength{\parskip}{0pt}
  \setlength{\parsep}{0pt}
    \item
        \begin{tabular*}{6.9in}{l@{\extracolsep{\fill}}r}
            \textbf{Buenos Aires Technological Institute (ITBA):} Mechanical Engineering. & 2013 - 2020 \\
        \end{tabular*}
            \begin{itemize}
                  \setlength{\itemsep}{1.75pt}
                  \setlength{\parskip}{0pt}
                  \setlength{\parsep}{0pt}
                    \item GPA: 6.91/10. Workload equivalent to 300 ECTS.  \\
               \end{itemize}

    \item
        \begin{tabular*}{6.9in}{l@{\extracolsep{\fill}}r}
            \textbf{Royal Institute of Technology (KTH) - Sweden:} Exchange student. & January 2018 - June 2018 \\
        \end{tabular*}
            \begin{itemize}
                  \setlength{\itemsep}{1.75pt}
                  \setlength{\parskip}{0pt}
                  \setlength{\parsep}{0pt}
                    \item Courses taken: MJ2426, MJ2443, MJ2246 (Rocket Propulsion), AF2024 (FEM in Analysis and Design).  \\
               \end{itemize}    

    \item 
        \begin{tabular*}{6.9in}{l@{\extracolsep{\fill}}r}
            \textbf{Exploring the Physics of Planetary Enviroments, KTH, Sweden.}  & August 2019\\
        \end{tabular*}
        \begin{tabular}{m{16cm} c}
        Summer course where 25 international students took part in both lectures and workshops with group tasks on specific planetary environments exploration techniques such as numerical simulations and analysis of telescope and spacecraft data. \\ More information at: https://www.kth.se/spp/education/summer-course-2019
        \end{tabular}
        
            \item
        \begin{tabular*}{6.9in}{l@{\extracolsep{\fill}}r}
            \textbf{Space Mission Design and Operations (EE585x), EPFLx, edX.org.}  & March 2020 - May 2020 \\
        \end{tabular*}
        \begin{tabular}{m{16cm} c}
        Online course offered by Ecole Polytechnique Federale de Lausanne. The course focused on conceptual understanding of space mechanics, maneuvers, propulsion and control systems used in all spacecraft. \\ Certificate: https://courses.edx.org/certificates/ce22565ecdd94908b52edb34dbddb73a
        \end{tabular}
        
        
\end{itemize}

{\Large \textbf{Languages}}
      \vspace{-1.5mm}
    
    \begin{itemize}
      \setlength{\itemsep}{3pt}
      \setlength{\parskip}{0pt}
      \setlength{\parsep}{0pt}
        \item
        \begin{tabular*}{6.9in}{l@{\extracolsep{\fill}}r}
            \textbf{Spanish:} Native speaker.\\ 
        \end{tabular*}
        \item
        \begin{tabular*}{6.9in}{l@{\extracolsep{\fill}}r}
            \textbf{English:} CEFR C2. \\
        \end{tabular*}
                \begin{itemize}
                  \vspace{-1mm}
                    \item Cambridge University, Certificate of Proficiency in English, Grade: 75/100 (B). \hfill December 2011\\
                \end{itemize}
                \begin{itemize}
                    \vspace{-3mm}
                    \item TOEFL iBT, Grade: 113/120. 
                    \hfill November 2019\\
                \end{itemize}
    \end{itemize}
      \vspace{-1mm}

{\Large \textbf{Technologies}}
      \vspace{-1.5mm}
    
    \begin{itemize}
      \setlength{\itemsep}{3pt}
      \setlength{\parskip}{0pt}
      \setlength{\parsep}{0pt}
    
        \item
        \begin{tabular*}{6.9in}{l@{\extracolsep{\fill}}r}
            \textbf{Ofimatic Tools:} Microsoft Office, \LaTeX.\\
        \end{tabular*}
        \item
        \begin{tabular*}{6.9in}{l@{\extracolsep{\fill}}r}
            \textbf{CAD Tools:} CATIA v5, GrabCAD.\\
        \end{tabular*}
        \item
        \begin{tabular*}{6.9in}{l@{\extracolsep{\fill}}r}
            \textbf{Programming:} Matlab, Python, Jupyter Notebook.\\
        \end{tabular*}
    \end{itemize}


{\Large \textbf{Work \& Volunteering Experiences}}
   \vspace{-1.5mm}
    \begin{itemize}
      \setlength{\itemsep}{3pt}
      \setlength{\parskip}{0pt}
      \setlength{\parsep}{0pt}

              \item
        \begin{tabular*}{6.9in}{l@{\extracolsep{\fill}}r}
            \textbf{LIA Aerospace:} Trajectory Engineer. & May 2020 - Present\\
        \end{tabular*}
        \begin{tabular}{m{16cm} c}
            LIA Aerospace is developing a launch vehicle to deploy small satellites to LEO. Main responsibilities are: 
                \end{tabular}
             \begin{itemize}
                  \setlength{\itemsep}{2pt}
                  \setlength{\parskip}{0pt}
                  \setlength{\parsep}{0pt}
                    \item Analysis of possible launch sites and orbital destinations.  \\
                    \item Create a 6-DOF simulation of the vehicle's trajectory from launch to satellite deployment. \\
            \end{itemize}     
        More information at: http://lia-aerospace.com/
              \item
        \begin{tabular*}{6.9in}{l@{\extracolsep{\fill}}r}
            \textbf{Sweden Alumni Network Argentina:} Chairperson. & January 2020 - Present\\
        \end{tabular*}
        \begin{tabular}{m{16cm} c}
            Sweden alumni networks around the world are open for anyone who have studied or carried out research in Sweden. The goal of the network is to provide a platform for alumni to stay in touch with Sweden and each other. One of the founding members of a 50+ members network. \\ More information at: https://alumni.si.se/alumni-networks/
        \end{tabular}

              \item
        \begin{tabular*}{6.9in}{l@{\extracolsep{\fill}}r}
            \textbf{Tenaris, Quality Assurance:} Summer intern. & January 2019 – March 2019\\
        \end{tabular*}
        \begin{tabular}{m{16cm} c}
            TenarisSiderca Cold Drawn and Components Center. Main responsabilities were:
        \end{tabular}
                 \begin{itemize}
                      \setlength{\itemsep}{2pt}
                      \setlength{\parskip}{0pt}
                      \setlength{\parsep}{0pt}
                        \item Validation of cold drawn and heat treatment processes as per IATF 16949:2016 requirements.  \\
                        \item Plan and conduct Measurement System Analysis (MSA) studies following AIAG guidelines. \\
                        \item Assist Product Engineering departament in the later phases of product development.  \\
                \end{itemize}     

       \item
        \begin{tabular*}{6.9in}{l@{\extracolsep{\fill}}r}
            \textbf{{Mathematics and physics teacher.}} & June 2015 – December 2019\\
        \end{tabular*}
        \begin{tabular}{m{16cm} c}
       I give classes to high-school and early university students at my home in order to assist them with both physics and mathematics learning. This initiative grew as a response to my interest in science education. 
        \end{tabular}

\end{itemize}

\hfill\textit{Last modified in August 2020.}
\end{document}
