% CV - Pablo Agustin Lobo 

% (c) 2018 Pablo Agustin Lobo 
%
\documentclass[a4paper,10pt]{article}
\usepackage{ucs}
\usepackage{multicol}
\usepackage{ragged2e}
\usepackage{lipsum}
\usepackage{array}
\usepackage[utf8x]{inputenc}
\usepackage[margin=0.5in]{geometry}
\raggedbottom
\raggedright
\setlength{\tabcolsep}{0in}
\pagestyle{empty}
\begin{document}

{\LARGE \textbf{Pablo Agustín Lobo} }\\
\medskip \hrule height 1pt \smallskip

\begin{tabular*}{7.25in}{l@{\extracolsep{\fill}}r}
& \textit{plobo@itba.edu.ar · https://www.linkedin.com/in/pablo-a-lobo/ · +(54911) 5116 8801}
\end{tabular*}

{\Large \textbf{Información Personal}}
  \vspace{-2.5mm}
    \begin{itemize}
      \setlength{\itemsep}{3pt}
      \setlength{\parskip}{0pt}
      \setlength{\parsep}{0pt}
      \begin{multicols}{2}
        \item \textbf{DNI:} 38069162.
        \item \textbf{Fecha de nacimiento:} 17/02/1994.
        \item \textbf{Lugar de nacimiento:} CABA, Argentina.
        \item \textbf{Nacionalidad:} Argentino.
        \item \textbf{Ciudadanía:} Argentina, España.
        \item \textbf{Estado civil:} Soltero.
      \end{multicols}
    \end{itemize}
  \vspace{-1mm}

\medskip

{\Large \textbf{Educación}}
  \vspace{-1.75mm}
\begin{itemize}
  \setlength{\itemsep}{2pt}
  \setlength{\parskip}{0pt}
  \setlength{\parsep}{0pt}
    \item
        \begin{tabular*}{6.9in}{l@{\extracolsep{\fill}}r}
            \textbf{Instituto Tecnológico de Buenos Aires (ITBA):} Ingeniería Mecánica. & 2013 - 2020 \\
        \end{tabular*}
            \begin{itemize}
                  \setlength{\itemsep}{1.75pt}
                  \setlength{\parskip}{0pt}
                  \setlength{\parsep}{0pt}
                    \item 97.5\% progreso. Fecha de graduación: Junio 2020.  \\
               \end{itemize}
    \item
        \begin{tabular*}{6.9in}{l@{\extracolsep{\fill}}r}
            \textbf{Royal Institute of Technology (KTH) - Suecia:} Alumno de intercambio. & Enero 2018 - Junio 2018 \\
        \end{tabular*}
                \begin{itemize}
                  \setlength{\itemsep}{1.75pt}
                  \setlength{\parskip}{0pt}
                  \setlength{\parsep}{0pt}
                    \item AF2024 - Finite Element Methods in Analysis and Design \\
                    \item MJ2246 - Rocket Propulsion \\
                    \item MJ2426 - Applied Heat and Power Technology \\
                    \item MJ2443 - Heating, Cooling and Indoor Climate \\
                \end{itemize}
    \item 
        \begin{tabular*}{6.9in}{l@{\extracolsep{\fill}}r}
            \textbf{Exploring the Physics of Planetary Enviroments (KTH) - Suecia.}  & Agosto 2019\\
        \end{tabular*}
        \begin{tabular}{m{16cm} c}
        Curso de verano donde 25 alumnos internacionales participaron de clases y talleres con tareas grupales basadas en la exploración de ambientes planetarios tales como simulaciones numéricas y análisis de datos obtenidos mediante telescopios y satélites. \\
        Más información en: https://www.kth.se/spp/education/summer-course-2019
        \end{tabular}
        
        \item
        \begin{tabular*}{6.9in}{l@{\extracolsep{\fill}}r}
            \textbf{Space Mission Design and Operations (EE585x), EPFLx, edX.org.}  & Marzo 2020 - Mayo 2020 \\
        \end{tabular*}
        \begin{tabular}{m{16cm} c}
        Curso online ofrecido por el Ecole Polytechnique Federale de Lausanne. El curso está enfocado en la comprensión conceptual de la astrodinámica, maniobras, propulsión y sistemas de control utilizados en naves espaciales. Las clases también incluyen ejemplos de los distintos desafíos que se presentan a la hora de utilizar el ámbito espacial con objetivos científicos y comerciales. \\ Certificado: https://courses.edx.org/certificates/ce22565ecdd94908b52edb34dbddb73a
        \end{tabular}
        
\end{itemize}

{\Large \textbf{Idiomas}}
      \vspace{-1.5mm}
    
    \begin{itemize}
      \setlength{\itemsep}{3pt}
      \setlength{\parskip}{0pt}
      \setlength{\parsep}{0pt}
        \item
        \begin{tabular*}{6.9in}{l@{\extracolsep{\fill}}r}
            \textbf{Español:} Nativo.\\ 
        \end{tabular*}
        \item
        \begin{tabular*}{6.9in}{l@{\extracolsep{\fill}}r}
            \textbf{Inglés:} CEFR C2. \\
        \end{tabular*}
                \begin{itemize}
                  \vspace{-1mm}
                    \item Cambridge University, Certificate of Proficiency in English, Grade: 75/100 (B). \hfill Diciembre 2011\\
                \end{itemize}
                \begin{itemize}
                    \vspace{-3mm}
                    \item TOEFL iBT, Grade: 113/120. 
                    \hfill Noviembre 2019\\
                \end{itemize}
    \end{itemize}
      \vspace{-1mm}

{\Large \textbf{Tecnologías}}
      \vspace{-1.5mm}
    
    \begin{itemize}
      \setlength{\itemsep}{3pt}
      \setlength{\parskip}{0pt}
      \setlength{\parsep}{0pt}
    
        \item
        \begin{tabular*}{6.9in}{l@{\extracolsep{\fill}}r}
            \textbf{Ofimática:} Microsoft Office Word/PowerPoint/Excel (Certificados oficiales),  \LaTeX.\\
        \end{tabular*}
        \item
        \begin{tabular*}{6.9in}{l@{\extracolsep{\fill}}r}
            \textbf{Software matemático:} Matlab, EES.\\
        \end{tabular*}
        \item
        \begin{tabular*}{6.9in}{l@{\extracolsep{\fill}}r}
            \textbf{Herramientas CAD:} CATIA v5, SolidWorks, GrabCAD.\\
        \end{tabular*}
        \item
        \begin{tabular*}{6.9in}{l@{\extracolsep{\fill}}r}
            \textbf{Análisis de elementos finitos:} Abaqus.\\
        \end{tabular*}
    \end{itemize}


{\Large \textbf{Experiencia}}
   \vspace{-1.5mm}
    \begin{itemize}
      \setlength{\itemsep}{3pt}
      \setlength{\parskip}{0pt}
      \setlength{\parsep}{0pt}
             \item
        \begin{tabular*}{6.9in}{l@{\extracolsep{\fill}}r}
            \textbf{Sweden Alumni Network Argentina: }Presidente. & Enero 2020 - Presente\\
        \end{tabular*}
        \begin{tabular}{m{16cm} c}
            Las redes de ex-alumnos de Suecia alrededor del mundo se encuentran abiertas para quienes hayan estudiado o realizado tareas de investigación en Suecia. El objetivo de las mismas es proveer una plataforma para que los alumnos se mantengan en contacto con Suecia y entre ellos. Parte del equipo fundador de una red con 50+ miembros. Más información en: https://alumni.si.se/alumni-networks/
        \end{tabular}
      
              \item
        \begin{tabular*}{6.9in}{l@{\extracolsep{\fill}}r}
            \textbf{{Tenaris, Aseguramiento de Calidad:} Pasante de verano.} & Enero 2019 – Marzo 2019.\\
        \end{tabular*}
        \begin{tabular}{m{16cm} c}
            TenarisSiderca, Trefila en Frío y Centro de Componentes. Mis principales responsabilidades fueron:
        \end{tabular}
                 \begin{itemize}
                      \setlength{\itemsep}{2.2pt}
                      \setlength{\parskip}{0pt}
                      \setlength{\parsep}{0pt}
                        \item Validación de los procesos de trefilado y tratamiento térmico según la norma IATF 16949:2016.\\
                        \item Planificación y ejecución de estudios de Measurement System Analysis (MSA) según la norma AIAG. \\
                        \item Revisión de documentación PPAP en colaboración con el departamento de Ingeniería de Producto. \\
                        \item Brindar asistencia al departamento de Ingeniería de Producto en la calificación de un producto para el comienzo de su producción.  \\
                \end{itemize}     

        \item
        \begin{tabular*}{6.9in}{l@{\extracolsep{\fill}}r}
            \textbf{{Diseño \& construcción de una impresora 3D.}} & Abril 2016 – Diciembre 2016\\
        \end{tabular*}
        \begin{tabular}{m{16cm} c}
        Trabajo voluntario. Parte de un grupo de 4 estudiantes del ITBA. Mi rol en el proyecto fue definir y dibujar en CAD una impresora que cumpla con los requerimientos del proyecto.
        \end{tabular}

       \item
        \begin{tabular*}{6.9in}{l@{\extracolsep{\fill}}r}
            \textbf{{Clases particulares: matemática y física.}} & Junio 2015 – Diciembre 2019\\
        \end{tabular*}
        \begin{tabular}{m{16cm} c}
        Doy clases particulares en mi casa a estudiantes secundarios y universitarios. Esta iniciativa surgió como respuesta a mi interés en la educación de ciencias básicas. 
        \end{tabular}

\end{itemize}

\hfill\textit{Última modificación: Junio 2020.}
\end{document}